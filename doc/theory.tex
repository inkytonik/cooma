\documentclass[11pt]{article}

\usepackage{amsmath}
\usepackage{amssymb}
\usepackage[T1]{fontenc}
\usepackage{fullpage}
\usepackage{semantic}
\usepackage{stmaryrd}

\setlength{\parindent}{0mm}
\setlength{\parskip}{3mm}

\newcommand{\kw}[1]{\mathtt{#1}}

\begin{document}

\section{Notation}

\newcommand{\fresh}[1]{\mathrm{fresh}\;#1}
\newcommand{\cons}[2]{\mathit{#1 :: #2}}
\newcommand{\map}[2]{\mathit{map}(#1, #2)}
\newcommand{\rep}[1]{\overline{#1}}

\newcommand{\arge}[2]{#1 : #2}
\newcommand{\flde}[2]{#1 = #2}
\newcommand{\flds}[2]{\rep{\flde{#1}{#2}}}

\[
\begin{array}{ll}
\rep{x}          & \text{sequence of zero or more $x_0, x_1, \ldots, x_n$} \\
x_h              & \text{head of $\rep{x}$} \\
x_h, t_h         & \text{components of first element of $\rep{\arge{x}{t}}$} \\
\rep{x}_t        & \text{tail elements of $\rep{x}$} \\
\cons{h}{t}      & \text{sequence with head $h$ and tail $t$} \\
\map{f}{\rep{x}} & \text{map operation $f$ over elements of $\rep{x}$} \\
\fresh{x}        & \text{$x$ is a unique identifier} \\
\end{array}
\]

\section{Cooma}

\newcommand{\ande}[2]{#1\;\&\;#2}
\newcommand{\appe}[2]{#1(#2)}
\newcommand{\blke}[1]{\{ #1 \}}
\newcommand{\fune}[2]{\kw{fun}\;#1\;.\;#2}
\newcommand{\nume}[1]{#1}
\newcommand{\rowe}[2]{\{ \flds{#1}{#2} \}}
\newcommand{\sele}[2]{#1 . #2}
\newcommand{\stre}[1]{#1}
\newcommand{\vare}[1]{#1}

\newcommand{\fd}[4]{#1(\rep{#2 : #3}) = #4}
\newcommand{\fundef}[2]{\rep{\kw{def}\;#1}\;#2}
\newcommand{\valdef}[3]{\kw{val}\;#1 = #2\;#3}
\newcommand{\return}[1]{#1}

\newcommand{\funty}[2]{#1 -> #2}
\newcommand{\intty}{\kw{Int}}
\newcommand{\rowty}[2]{\{ \rep{#1 : #2} \}}
\newcommand{\strty}{\kw{String}}

\[
\begin{array}{lcl}
n & \in & \text{integer} \\
s & \in & \text{string} \\
x & \in & \text{identifier} \\
\end{array}
\]
\[
\begin{array}{rlcll}
\text{expression}
  & e & ::= & \nume{n}                      & \text{integer constant} \\
  &   & |   & \stre{s}                      & \text{string constant} \\
  &   & |   & \vare{x}                      & \text{variable} \\
  &   & |   & \fune{\rep{\arge{x}{t}}}{e}   & \text{function abstraction} \\
  &   & |   & \appe{e}{\rep{e}}             & \text{function application} \\
  &   & |   & \blke{b}                      & \text{block} \\
  &   & |   & \rowe{x}{e}                   & \text{row} \\
  &   & |   & \sele{e}{x}                   & \text{row selection} \\
  &   & |   & \ande{e}{e}                   & \text{row concatenation} \\
\\
\text{block expression}
  & b & ::= & \valdef{x}{e}{b}              & \text{value definition} \\
  &   & |   & \fundef{\fd{x}{x}{t}{e}}{b}   & \text{function definitions} \\
  &   & |   & \return{e}                    & \text{return} \\
\\
\text{type}
  & t & ::= & \intty       & \text{integer} \\
  &   & |   & \strty       & \text{string} \\
  &   & |   & \funty{t}{t} & \text{function} \\
  &   & |   & \rowty{x}{t} & \text{row} \\
\end{array}
\]

Notes:
\begin{itemize}
\item Functions and function calls take one or more arguments.
\item Blocks are ``lets'' that contain value definitions and sections of mutually-recursive function definitions.
\item Rows are collections of named fields. Rows can be selected from and concatenated.
\end{itemize}

\newpage

\section{Intermediate Representation}

\newcommand{\appc}[2]{#1\;#2}
\newcommand{\appf}[3]{#1\;#2\;#3}
\newcommand{\letc}[4]{\kw{letc}\;#1\;#2 = #3\;\kw{in}\;#4}

\newcommand{\defterm}[4]{#1\;#2\;#3 = #4}
\newcommand{\letf}[2]{\kw{letf}\;\rep{#1}\;\kw{in}\;#2}
\newcommand{\letff}[2]{\kw{letf}\;#1\;\kw{in}\;#2}

\newcommand{\letv}[3]{\kw{letv}\;#1 = #2\;\kw{in}\;#3}

\newcommand{\funv}[3]{\kw{fun}\;#1\;#2\;.\;#3}
\newcommand{\intv}[1]{#1}
\newcommand{\prmv}[2]{#1(#2)}
\newcommand{\rowv}[2]{\{ \rep{#1 = #2} \}}
\newcommand{\rowvs}[2]{\{ #1 \}}
\newcommand{\strv}[1]{#1}

\newcommand{\andv}[2]{\kw{concat}\;#1\;#2}
\newcommand{\argv}[1]{\kw{argument}\;#1}
\newcommand{\capv}[2]{\kw{capability}\;#1\;#2}
\newcommand{\selv}[2]{\kw{select}\;#1\;#2}

\[
\begin{array}{lcl}
n & \in & \text{integer} \\
s & \in & \text{string} \\
f, k, x & \in & \text{identifier} \\
p & \in & \text{primitive} \\
\end{array}
\]

\[
\begin{array}{rlcll}
\text{term}
  & z & ::= & \letc{k}{x}{z}{z}              & \text{bind continuation} \\
  &   & |   & \letf{\defterm{f}{k}{x}{z}}{z} & \text{bind functions} \\
  &   & |   & \letv{x}{v}{z}                 & \text{bind value} \\
  &   & |   & \appc{k}{x}                    & \text{apply continuation} \\
  &   & |   & \appf{f}{k}{x}                 & \text{apply function} \\
\\
\text{value}
  & v & ::= & \intv{n}           & \text{integer}  \\
  &   & |   & \strv{s}           & \text{string} \\
  &   & |   & \funv{k}{x}{z}     & \text{function} \\
  &   & |   & \rowv{x}{x}        & \text{row} \\
  &   & |   & \prmv{p}{\rep{x}}  & \text{primitive call} \\
\end{array}
\]

Notes:
\begin{itemize}
\item Terms are from Kennedy ``Compiling with Continuations, Continued'', except $\kw{letf}$ binds multiple mutually-recursive functions, instead of a single function.
\item Values are standard, plus rows and primitive calls.
\item Primitives encode basic operations, including:
\begin{itemize}
\item row selection and concatenation
\item get the n-th command-line argument as a string
\item check a capability and, if the capability is available (e.g., the user has permission), return the capability value as a row that contains the available operations
\item capability operations such as read or write a file
\end{itemize}
\end{itemize}

\newpage

\section{Compilation}

\newcommand{\cmp}[2]{\llbracket #2 \rrbracket_#1}
\newcommand{\cmpn}[1]{\cmp{#1}{\cdot}}
\newcommand{\cmpin}[3]{\cmp{#1}{#2}\;#3}
\newcommand{\cmpk}[2]{\cmp{#1}{#2}\;\kappa}

\newcommand{\tcmp}[2]{\llparenthesis #2 \rrparenthesis_#1}
\newcommand{\tcmpn}[1]{\tcmp{#1}{\cdot}}
\newcommand{\tcmpin}[3]{\tcmp{#1}{#2}\;#3}
\newcommand{\tcmpk}[2]{\tcmp{#1}{#2}\;k}

The compilation scheme follows the lead of Kennedy ``Compiling with Continuations, Continued''.

The compilation functions come in standard form $\cmpn{e}$ and sometimes also a tail-call form $\tcmpn{e}$, where the subscript $e$, $b$, etc denotes the form of the code being compiled: expression, block, etc.

Apart from the code to compile, the standard form takes an argument $\kappa$ that is a function from an identifier to a compiled term.
I.e., $\kappa$ is a code generation continuation that produces the rest of the code, given the name of the variable to which the result of the current compiled code has been bound.
In other words, if the variable $x$ holds the result computed by the code we just compiled, then $\kappa(x)$ computes the complete code.

The tail form is similar, except it is used when the continuation is known at compile time, so an efficient direct call can be generated.
Thus, instead of $\kappa$, the extra argument to the tail form is the name of the relevant continuation variable $k$.
We just generate a call to this continuation, instead of using $\kappa$ to generate the code.
In other words, if the variable $x$ holds the result computed by the code we just compiled, then we can pass it directly to the continuation using $\appc{k}{x}$.

\subsection*{Top-level}

We distinguish two cases of top-level compilation: one for a whole program that is intended to be run from the command line, and one for an expression entered in the REPL.

In the program case, $\cmpn{p}$ curries functions and compiles capability arguments to access the command line argument strings and create capabilities.
The parameter $n$ keeps count of the current argument number, so we compile a program $e$ with $\cmpin{p}{e}{0}$.
\[
    \cmpn{p} : e -> n -> z
\]
\[
\begin{array}{rcll}
\cmpin{p}{\fune{\arge{a}{t}}{e}}{n} & = &
  \letv{x}{\prmv{\kw{argument}}{n}}{\letv{a}{\prmv{\kw{capability}}{t,x}}{\cmpin{p}{e}{(n + 1)}}} \\
  & &
  \text{$t$ is a capability}, \fresh{x} \\
\end{array}
\]

Other than capability arguments, string program arguments are passed through from the command line.
\[
\begin{array}{rcll}
\cmpin{p}{\fune{\arge{a}{\kw{String}}}{e}}{n} & = &
  \letv{a}{\prmv{\kw{argument}}{n}}{\cmpin{p}{e}{(n + 1)}}
  \\
\end{array}
\]
All other program argument types are in error.

If the program is not a function, it is compiled in a normal fashion using the special continuation variable $\kw{halt}$ that terminates program execution when called.
\[
\begin{array}{rcll}
\cmpin{p}{e}{n} & = &
  \tcmpin{e}{e}{\kw{halt}} &
  \text{otherwise}
\end{array}
\]
In the REPL case, there is no special handling of command line arguments, so we compile each expression as in the final case of $\cmpn{p}$.

\subsection*{Expressions}

\[
    \cmpn{e} : e -> (x -> z) -> z
\]
\[
    \tcmpn{e} : e -> x -> z
\]
Constants compile into bound values in the current continuation (only numbers shown).
\[
\begin{array}{rcll}
\cmpk{e}{\nume{n}} & = &
  \letv{x}{n}{\kappa(x)}   & \fresh{x} \medskip \\

\tcmpk{e}{\nume{n}} & = &
  \letv{x}{n}{\appc{k}{x}} & \fresh{x} \\
\end{array}
\]
Variables compile into their use in the current continuation.
\[
\begin{array}{rcll}
\cmpk{e}{\vare{x}} & = &
  \kappa(x) \medskip \\

\tcmpk{e}{\vare{x}} & = &
  \appc{k}{x} \\
\end{array}
\]
Functions are curried and compiled one argument at a time by $\cmpn{f}$ or $\tcmpn{f}$.
\[
\begin{array}{rcll}
\cmpk{e}{\fune{\arge{x}{t}}{e}} & = &
  \cmpk{f}{x,t,e}
  \\
\cmpk{e}{\fune{\rep{\arge{x}{t}}}{e}} & = &
  \cmpk{f}{x_h,t_h,\fune{\rep{\arge{x}{t}}_t}{e}} \medskip \\

\tcmpk{e}{\fune{\arge{x}{t}}{e}} & = &
  \tcmpk{f}{x,t,e}
  \\
\tcmpk{e}{\fune{\rep{\arge{x}{t}}}{e}} & = &
  \tcmpk{f}{x_h,t_h,\fune{\rep{\arge{x}{t}}_t}{e}}
\end{array}
\]
Applications are curried.
An application to a single argument is compiled by compiling the function $f$ to $y$, the argument $e$ to $z$, capturing the current continuation in $k$ and generating the function call to continue with $k$.
\[
\begin{array}{rcll}
\cmpk{e}{\appe{f}{e}} & = &
  \cmpin{e}{f}{(y => \cmpin{e}{e}{(z => \letc{k}{x}{\kappa(x)}{\appf{y}{k}{z}})})} &
  \fresh{k,x} \\
\cmpk{e}{\appe{f}{\rep{e}}} & = &
  \cmpk{e}{\appe{\appe{f}{e_h}}{\rep{e}_t}} \medskip \\

\tcmpk{e}{\appe{f}{e}} & = &
  \cmpin{e}{f}{(y => \cmpin{e}{e}{(z => \appf{y}{k}{z})})} \\
\tcmpk{e}{\appe{f}{\rep{e}}} & = &
  \tcmpk{e}{\appe{\appe{f}{e_h}}{\rep{e}_t}} \\
\end{array}
\]
Block expressions are compiled by $\cmpn{b}$ or $\tcmpn{b}$.
\[
\begin{array}{rcll}
\cmpk{e}{\blke{b}} & = &
  \cmpin{b}{b}{\kappa} \medskip \\

\tcmpk{e}{\blke{b}} & = &
  \tcmpin{b}{b}{k} \\
\end{array}
\]
Rows have their field definitions compiled by $\cmpn{r}$.
The resulting sequence of field values $z$ is bound in a new row $x$ for use in the continuation.
\[
\begin{array}{rcll}
\cmpk{e}{\rowe{f}{e}} & = &
  \cmpin{r}{\flds{f}{e}}{(z => \letv{x}{\rowvs{z}}{\kappa(x)})} \medskip \\

\tcmpk{e}{\rowe{f}{e}} & = &
  \cmpin{r}{\flds{f}{e}}{(z => \letv{x}{\rowvs{z}}{\appc{k}{x}})} \\
\end{array}
\]
Row operations compile into primitives that use the compiled row arguments $y$ and $z$.
\[
\begin{array}{rcll}
\cmpk{e}{\sele{r}{f}} & = &
  \cmpin{e}{r}{(z => \letv{x}{\selv{z}{f}}{\kappa(x)})} \\
\cmpk{e}{\ande{l}{r}} & = &
  \cmpin{e}{l}{(y => \cmpin{e}{r}{(z => \letv{x}{\andv{y}{z}}{\kappa(x)})})} \medskip \\

\tcmpk{e}{\sele{r}{f}} & = &
  \cmpin{e}{r}{(z => \letv{x}{\selv{z}{f}}{\appc{k}{x}})} \\
\cmpk{e}{\ande{l}{r}} & = &
  \cmpin{e}{l}{(y => \cmpin{e}{r}{(z => \letv{x}{\andv{y}{z}}{\appc{k}{x}})})} \\
\end{array}
\]

\subsection*{Blocks}

Blocks compile into nested lets that mirror the block expression structure.

A function definition group turns into an analogous $\kw{letf}$ where each definition has been compiled by $\cmpn{d}$, and whose body is the compiled nested block expression.

A value definition compiles into its constituent expression and a continuation $j$ to get to the nested block expression.
We use the tail form of the compiler on $e$ since we know $j$ statically.

A return expression is trivially compiled using $\cmpn{e}$ or $\tcmpn{e}$.
\[
    \cmpn{b} : b -> (x -> z) -> z
\]
\[
\begin{array}{rcll}
\cmpk{b}{\fundef{f}{b}} & = &
  \letff{\map{\cmpn{d}}{\rep{f}}}{\cmpk{b}{b}} \\
\cmpk{b}{\valdef{x}{e}{b}} & = &
  \letc{j}{x}{\cmpk{b}{b}}{\tcmpin{e}{e}{j}} & \fresh{j} \\
\cmpk{b}{\return{e}} & = &
  \cmpk{e}{e} \medskip \\

\tcmpk{b}{\fundef{f}{b}} & = &
  \letff{\map{\cmpn{d}}{\rep{f}}}{\tcmpk{b}{b}} \\
\tcmpk{b}{\valdef{x}{e}{b}} & = &
  \letc{j}{x}{\tcmpk{b}{b}}{\tcmpin{e}{e}{j}} & \fresh{j} \\
\cmpk{b}{\return{e}} & = &
  \tcmpk{e}{e} \\
\end{array}
\]

\subsection*{Function definitions}

A function definition compiles into a binding for use in a $\kw{letf}$ where a continuation argument $k$ has been added and the body $e$ has been compiled to $z$.
The body is either compiled directly if the function only has one argument, or compiled as a function that takes the remaining arguments otherwise.
\[
    \cmpn{d} : x -> \rep{\arge{x}{t}} -> e -> (x -> z) -> z
\]
\[
\begin{array}{rcll}
\cmpk{d}{\fd{f}{x}{t}{e}} & = &
  \defterm{f}{k}{x}{z} &
  \fresh{k} \\
\text{where $z$} & = &
  \tcmpk{e}{e} & \text{if $\rep{\arge{x}{t}}$ is a singleton} \\
 & = &
  \tcmpk{e}{\fune{\rep{\arge{x}{t}}_t}{e}} & \text{otherwise} \\
\end{array}
\]

\subsection*{Functions}

Functions that have a capability argument are compiled into an invocation of the appropriate capability primitive (here denoted $C$).
The primitive is responsible for checking the capability permission and, if the check passes, returning a row that contains the capability operations.
The capability row is bound to the function argument $x$ for use in the compiled function body $e$.

Functions that have a non-capability argument are compiled to a simple $\kw{fun}$.
\[
    \cmpn{f} : x -> t -> e -> (x -> z) -> z
\]
\[
\begin{array}{rclll}
\cmpk{f}{x, t, e} & = &
  \letv{f}{(\funv{j}{y}{\letv{x}{\prmv{C}{y}}{\tcmpin{e}{e}{j}}})}{\kappa(f)} &
  \text{$t$ is a capability} \\
  & & &  \fresh{f, j, y} \\
\cmpk{f}{x, t, e} & = &
  \letv{f}{\funv{j}{x}{\tcmpin{e}{e}{j}}}{\kappa(f)} & \text{otherwise}, \fresh{f, j} \medskip \\

\tcmpk{f}{x, t, e} & = &
  \letv{f}{(\funv{j}{y}{\letv{x}{\prmv{C}{y}}{\tcmpin{e}{e}{j}}})}{\appc{k}{f}} &
  \text{$t$ is a capability type} \\
  & & &  \fresh{f, j, y} \\
\tcmpk{f}{x, t, e} & = &
  \letv{f}{\funv{j}{x}{\tcmpin{e}{e}{j}}}{\appc{k}{f}} & \text{otherwise}, \fresh{f, j}
\end{array}
\]

\subsection*{Field definitions}

Field definitions $\flds{f}{e}$ are compiled by compiling their defining expressions and building up a sequence of corresponding field values $\flds{f}{v}$.
\[
    \cmpn{r} : \flds{f}{e} -> (\flds{f}{v} -> z) -> z
\]
\[
\begin{array}{rcll}
\cmpk{r}{\flds{f}{e}}
  & = & \kappa([]) & \text{if $\flds{f}{e}$ is empty} \\
  & = & \cmpin{e}{e_h}{(y =>
          \cmpin{r}{\flds{f}{e}_t}{(z =>
            \kappa(\cons{(\flde{f_h}{y})}{z}))})}
      & \text{otherwise} \\
\end{array}
\]

\newpage

\section{Interpretation}

\newcommand{\interp}[2]{#1 |- #2 \Downarrow}

\newcommand{\conf}[3]{\lambda #1 #2 . #3}
\newcommand{\conc}[2]{\lambda #1 . #2}

\newcommand{\clsf}[4]{\langle #1, \conf{#2}{#3}{#4} \rangle}
\newcommand{\clsc}[3]{\langle #1, \conc{#2}{#3} \rangle}
\newcommand{\clsfs}[5]{\langle #1, \rep{\lambda #2 #3 #4 . #5} \rangle}
\newcommand{\clsffs}[2]{\langle #1, \rep{#2} \rangle}

\newcommand{\extv}[3]{#1, #2 \mapsto #3}
\newcommand{\extc}[5]{\extv{#1}{#2}{\clsc{#3}{#4}{#5}}}
\newcommand{\extf}[6]{#1, \clsfs{#2}{#3}{#4}{#5}{#6}}
\newcommand{\extff}[2]{#1, \clsffs{#2}}

We describe interpretation of the intermediate representation using the relation $\interp{\rho}{z}$ that computes a runtime value given an environment $\rho$.

\subsection*{Runtime values}

Runtime values are basic values, function closures and rows.
\[
\begin{array}{rlcll}
\text{runtime value}
  & r & ::= & \intv{n}              & \text{integer}  \\
  &   & |   & \strv{s}              & \text{string} \\
  &   & |   & \clsf{\rho}{k}{x}{z}  & \text{closure} \\
  &   & |   & \rowv{x}{r}           & \text{row} \\
\end{array}
\]

\subsection*{Environments}

Environments record value bindings, continuation closures and function definitions  closures.
The latter share a single captured environment in which all of the definitions are bound.
\[
\begin{array}{rlcll}
\text{environment}
  & \rho & ::= & \bullet                       & \text{empty} \\
  &      & |   & \extv{\rho}{x}{r}             & \text{extend with value binding} \\
  &      & |   & \extc{\rho}{k}{\rho}{x}{z}    & \text{extend with continuation closure} \\
  &      & |   & \extf{\rho}{\rho}{f}{k}{x}{z} & \text{extend with function definitions closure}
\end{array}
\]
Environment lookup is $\rho(x)$.
In the function definitions case, the returned value is a single function closure $\clsf{\rho}{k}{x}{z}$ constructed from the function definition group closure.
\[
\begin{array}{rlll}
(\extv{\rho}{x}{r})(y) & = &
  r &
  \text{if $x = y$} \\
(\extc{\rho}{k}{\rho_1}{x}{z})(y) & = &
  \clsc{\rho_1}{x}{z} &
  \text{if $k = y$} \\
(\extf{\rho}{\rho_1}{f}{k}{x}{z})(y) & = &
  \clsf{\rho_1}{k_i}{x_i}{z_i} &
  \text{if $f_i = y$} \\
(\rho, \_)(y) & = &
  \rho(y) &
  \text{otherwise}
\end{array}
\]
This definition doesn't type properly since the second case returns a continuation closure and all of the other cases return a runtime value.
The lookup is two functions in the implementation.

\subsection*{Value interpretation}

\[
    \cmpn{v} : v -> \rho -> r
\]

Constants are their own interpretation.
\[
\begin{array}{rlll}
\cmp{v}{\intv{n}}{\rho} & = & n \\
\cmp{v}{\strv{s}}{\rho} & = & s \\
\end{array}
\]

A function value becomes a closure.
\[
\begin{array}{rlll}
\cmp{v}{\funv{k}{x}{z}}{\rho} & = & \clsf{\rho}{k}{x}{z} \\
\end{array}
\]

A row value becomes a runtime row value after the fields have been interpreted.
\[
\begin{array}{rlll}
\cmp{v}{\rowv{x}{y}}{\rho} & = \rowv{x}{\rho(y)} &
\end{array}
\]

Interpretation of primitives uses an external $\kw{eval}$ operation.
\[
\begin{array}{rlll}
\cmp{v}{\prmv{p}{\rep{x}}}{\rho} & = & \kw{eval}(p, \rep{x}, \rho)
\end{array}
\]

\subsection*{Rules (Lets)}

The let forms extend the environment with bindings in one of the three ways, then evaluate their body in the updated environment.

A $\kw{letv}$ updates the environment with a binding of the interpreted value.
\[
\inference[(LetV)]
{\interp{\extv{\rho}{x}{\cmp{v}{v}{\rho}}}{z}}
{\interp{\rho}{\letv{x}{v}{z}}}
\]

A $\kw{letc}$ updates the environment with a new continuation closure.
\[
\inference[(LetC)]
{\interp{\extc{\rho}{k}{\rho}{x}{z_1}}{z_2}}
{\interp{\rho}{\letc{k}{x}{z_1}{z_2}}}
\]

A $\kw{letf}$ updates the environment with a new function definitions closure.
\[
\inference[(LetF)]
{\rho_2 = \extff{\rho}{\rho_2}{d} & \interp{\rho_2}{z}}
{\interp{\rho}{\letf{d}{z}}}
\]

\subsection*{Rules (Calls)}

Evaluation terminates when the special $\kw{halt}$ continuation is called on $x$
The result of the evaluation is $\rho(x)$.
\[
\inference[(Halt)]
{}
{\interp{\rho}{\appc{\kw{halt}}{x}}}
\]

Otherwise, a continuation call gets the continuation closure from the environment and invokes the closure function body in the saved environment extended with the argument binding.
\[
\begin{array}{cl}
\inference[(AppC)]
{\interp{\extv{\rho_1}{y}{\rho(x)}}{z}}
{\interp{\rho}{\appc{k}{x}}}
&
\text{if $\rho(k) = \clsc{\rho_1}{y}{z}$}
\end{array}
\]

A function call is similar, except using a function closure and also binding the continuation argument.
\[
\begin{array}{cl}
\inference[(AppF)]
{\interp{\extv{\extv{\rho_1}{j}{\rho(k)}}{y}{\rho(x)}}{z}}
{\interp{\rho}{\appf{f}{k}{x}}}
&
\text{if $\rho(f) = \clsf{\rho_1}{j}{y}{z}$}
\end{array}
\]

\end{document}
